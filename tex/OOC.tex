\documentclass[a4paper, 12pt]{book}
\usepackage[colorlinks,linkcolor=blue,anchorcolor=blue,citecolor=green,CJKbookmarks=true]{hyperref}
\usepackage{fontspec,xltxtra,xunicode}
\usepackage{xeCJK}
\setCJKmainfont[BoldFont={STHeiti},ItalicFont={STKaiti}]{STSong}
\setmainfont{Times New Roman}
\setCJKsansfont[BoldFont={STHeiti}]{STXihei}
\setsansfont{Helvetica}
\setCJKmonofont{STFangsong}
\setmonofont{Courier New}
\usepackage{indent first}
\usepackage{color}
\usepackage[
			margin=0in,
			left=1.25in,
			right=1.25in,
			top=1in,
			bottom=1.5in
	]{geometry}
\usepackage[center,pagestyles]{titlesec}
\renewcommand\contentsname{目录}
\renewcommand\chaptername{第{\thechapter}章}
\newcommand{\sectionname}{节}
\renewcommand\bibname{参考文献}
\renewcommand{\figurename}{图}
\renewcommand{\tablename}{表}
\titleformat{\chapter}[block]{\center\Large\bf}{\chaptername}{20pt}{}
\titleformat{\section}[block]{\large\bf}{\thesection}{10pt}{}
\newpagestyle{mystyle}{
	\sethead[~\thechapter~~\chaptertitle~   ~\sectiontitle~][][\thepage]
	{\thepage}{}{第~\thesection~节~~~\sectiontitle}
	\setfoot[][][]
	{}{}{}
}
\pagestyle{mystyle}
\usepackage{listings}
\lstset{
	language=[ANSI]{C},
	basicstyle=\ttfamily\footnotesize,				% 基本字体的字号
	backgroundcolor=\color[RGB]{245,245,244},		% 背景色 需要 \usepackage{color} 
	breakatwhitespace=false,		% 断行只在空格处 
	breaklines=true,	% 自动断行 
	captionpos=b, 		% 标题位置 
	commentstyle=\slshape\color[RGB]{0,96,96},
	directivestyle=\color{blue},
	extendedchars=false,
	escapeinside=`` , % 添加注释,暂时离开 listings
	framerule=0pt,
	%identifierstyle=\color{red},
	keywordstyle=\color{blue},
	morekeywords={*,define,*,include...},
	numbers=left,						% 行号位置
	numbersep=5pt,					%行号与代码间距 
	%stepnumber=2,						%行号的显示步长 
	numberstyle=\tt\tiny\color[RGB]{0,192,192},			% 行号字体的字号 
	rulesepcolor=\color[RGB]{20,20,20},
	showspaces=false,					%显示空格 
	showstringspaces=false,				% 字符串中显示空格 
	showtabs=false,
	stringstyle=\slshape\color[RGB]{128,0,0},
	tabsize=4,
	title=\lstname
}
\usepackage{graphicx}
\XeTeXlinebreaklocale “zh”
\XeTeXlinebreakskip = 0pt plus 1pt minus 0.1pt 	%文章内中文自动换行

\linespread{1.2}	% 设置行距
\setlength{\parindent}{2em}
\setlength{\parskip}{1ex plus 0.5ex minus 0.2ex}

\title{使用ANSI-C进行面向对象编程}
\author{张盼}
\begin{document}

\maketitle

\frontmatter
\chapter{译者序}

看见这本书有两年了,一开始没注意,后来才发现成书到今天已经有20多年了,然而还没有中译本,于是萌生了翻译它的念头。

由于书的内容牵扯到很多不同的内容,所以翻译比较缓慢。第一次翻译的时候到第四章的时候,正好赶上期末考,于是就放下了,后来再次翻译的时候已经找不到原来翻译的前四章了。所以只好从头再来。

翻译的过程中也遇到不少问题。一方面因为成书年代较早,当时的系统和机器和现在的已经有了些许差异,而这在书中也有所体现,但是无需担心,我在碰到的地方都做了解释说明。另一方面由于此书面向unix平台,awk语言和X界面的那些代码需要重现的话,最好有个带X的GNU系统。
	%译者序
\chapter{前言}

\begin{flushright}
没有可以解决任何问题的编程技术\\
没有能只生成正确结果的编程语言\\
没有每个工程都从头开始的程序员\\
\end{flushright}


面向对象编程已经出现了十多年,它目前仍是解决问题的灵丹妙药。本质上,除了接 受了二十多年来的一
些好的编程法则外并没有什么新的东西带给我们。C++ 是一门新的语 言因为它是面向对象的,如果你不想使用或者不知道如何使用那么你不需要使用它,因为 普通的 C 就可以实现面向对象。虽然子程序的思想和计算机一样久远并且好的程序员总 是随身携带着他们的工具和库,但是只有面向对象才可以在不同项目间实现代码复用。

这本书不准备推崇面向对象或者批评传统的方式。我们准备以 ANSI-C 来发掘如何实 现面向对象,有哪些技术,为什么这些技术能帮我们解决更大的问题,如何利用它的一般 性以及更早的捕获异常。虽然我们会接触很多术语,如类、继承、实例、连接、方法、对 象、多态等等,但是我们将会剥去其魔幻的外表,使用我们熟悉的事物来表述他们。


我非常有意思的发现了 ANSI-C 其实是一门完全的面向对象的语言。如果想要和我分 享这份乐趣你需要非常熟悉它,至少也要对结构、指针、原型和函数指针。通过这本书, 你将遇到一个“新语言”——按照 Orwell 和韦氏词典对一门语言的解释,语言的设计目的 就是缩减思维的广度——而我会尽力证明,它不仅仅汇合了所有的那些你想汇聚到一起的 良好的编程原则。结果,你可以成为一个更熟练的 ANSI-C 程序员。


前六章建立 ANSI-C 做面向对象编程的基础。我们从一个抽象数据类型的信息隐藏技 术开始,然后加入基于动态连接的通用函数,再通过谨慎地扩充结构来继承代码。最后, 我们将上述所有放进一个类树中,来使代码更容易地维护。


编程需要规范。良好的编程更需要很多的规范、众多原则和标准以及确保正确无误的 防范措施。程序员使用工具,而优秀的程序员则制作工具来一劳永逸地处理那些重复的工 作。用 ANSI-C 的面向对象的编程需要相当大量的不变的代码——名称可能变化但结构不 变。因此,在第 7 章里我们搭建一个小小的预处理器,用来创建所需要的模板。它很像是 另一个方言式面向对象的语言。但是它不应该这样被看待,它剔除“方言”中枯燥无味的 东西,让我们专注于用更好的技术解决问题的创新。OOC 有非常好的可塑性:我们创造 了它,了解它,能够改变它,而且它可以如我们所愿的写 ANSI-C 代码。

余下章节继续深入讨论我们的技术。第 8 章加入动态类型检测来实现错误的早期捕 获。第 9 章讲我们通过使用自动初始化来防止另一类软件缺陷。第 10 章引入委托代理, 说明类和回调函数如何协作,比如去简化标准主程序的生成这样的常规任务。其他章节专 注于用类方法来堵塞内存泄漏,用一致的方法来存储和加载结构数据,和通过嵌套异常处 理系统的规范错误的恢复。

在最后一章,我们突破 ANSI-C 的限制,做了一个时髦的鼠标操作的计算器——先是 针对 curses 然后是针对 X Window 系统。这个例子极好地表明:即使是不得不应对外部 库和类树的风格,通过对象和类我们已然可以非常精致地进行设计和实现。

每一章都有总结,这些总结中我试图给随意浏览的读者一个梗概以及它对此后章节的 重要性。大多数的章节都有练习题,不过他们并不是正式的阐明性文字,因为我坚定的相 信读者应当自己实践。由于该技术是我们从无到有建立起来的,所以尽管有些例子应该能 够从中获益,但是我避免建立和使用庞大的类库。如果你想要真正地理解面向对象的编 程,首先掌握该技术并且在代码设计阶段考虑你的选择更为重要;而开发中依赖使用他人 的库应当在这稍后一点。

本书的一个重要部分是所附源码软盘2,——其上有一个 DOS 文件系统,包括一个用 来按照章节顺序来创建源码的简单 shell 脚本。还有一个 ReadMe 文件——在你执行 make 命令前要先查阅这个文件。使用一个工具如 diff 并且追踪根类和 OOC 报告在后续章节 的演化也是非常有帮助的。

这里展现的技术源自我对 C++ 的失望。当时我需要面向对象技术实现一个交互式 编程语言,但我意识到无法用 C++ 建立一个可移植的东西来。于是我转向我所了解的 ANSI-C,并且我完全能够做到要做的事情。我将这个些告诉组里的几个人,然后他们用同 样的方法完成了他们的工作。如果不是布赖恩 · 克尼翰(Brian Kernighan)以及我的出 版商翰斯 · 尼科拉斯(Hans-Joachim Niclas)、约翰 · 维特(John Wait)鼓励我出版这 些笔记(在适当的时候全新的展现一下),这个事情很可能就止于此,我的注解也就是一 时的时尚了。我感谢他们和所有帮助并且经历本书不断完善的人。最后但是并非不重要 的,感谢我的家庭——面向对象当然绝不可能代替餐桌上的面包。

1993 年 10 月于 Hollage

阿塞尔—托彼亚斯 · 斯莱内尔(Axel-Tobias Schreiner)	%原书前言

\mainmatter


\tableofcontents

\chapter{抽象数据类型-信息隐藏}
\section{数据类型}

数据类型是每一种编程语言必须具有的一部分。ANSI-C具有int,double和char等。程序员很少满足于编程语言提供的数据类型,并且编程语言一般提供了根据预定义数据类型来定义新的数据类型的能力。一种简单的途径是聚合形成例如数组、结构体或者联合等。指针,按照C.A.R. Hoare所言“我们永远不可能从之恢复”,允许我们指代和操作数据以一种没有界限的复杂度。


到底什么是数据类型?我们可以从不同的方面来看。一个数据类型是一个值的集合-char具有256个不同的值,int多一点;但两者都是间隔的,并且行为或多或少的类似于数学中的整数。double具有更多的值,但是它们肯定与数学中的实数行为不同。


可选的,我们可以定义一个新的数据类型作一些值的集合,并且赋予基于它们的操作。典型的,这些值是计算机可以表示的,并且操作或多或少的反映了可用的硬件指令。ANSI-C中的int就表现不是很好:在不同的机器之间值的集合可能会不一致,并且算数右移可能具有不同的表现。


更复杂的例子并不会进展得更好。一般地我们定义一个线性表得元素作为一个结构体

\begin{lstlisting}[language=C]
	typedef struct node {
		struct node * next;
		… information …
	} node;
\end{lstlisting}
\chapter{继承 代码重用和优化}
\section{一个父类——点类}
\section{父类的实现——Point类}
\section{继承——Circle类}
\section{绑定和继承}
\section{静态和动态绑定}
\section{可见性和访问函数}
\section{子类的实现--Circle类}

我们已经准备好实现circle类的完整表示,这里我们可以选择前边几节中我们喜欢的无论什么技术。面向对象规定我们需要一个构建子,或许还要一个析构子,\verb|Circle_draw()|,和一个类型描述Circle来绑定到一起。为了联系我们的方法,我们包含Circle.h并且添加如下行到4.1节中的switch中:

\begin{lstlisting}
case 'c':
	p = new(Circle, 1, 2, 3);
	break;
\end{lstlisting}

现在我们可以观察下面的测试程序的行为:

\begin{lstlisting}
$ circles p c
"." at 1,2
"." at 11,22
circle at 1,2 rad 3
circle at 11,22 rad 3
\end{lstlisting}

这个圆构建子接收到三个参数:首先是圆心的坐标,然后是半径。初始化点的部分是点构建子的工作。它取走\verb|new()|参数列表的一部分,圆构建子使用留下的参数列表,从之初始化半径。

一个子类构建子应当首先让父类构建子做把内存区域变成父类对象的初始化部分。一旦父类的构建子完成了,子类的构建子完成初始化并且把父类对象变成子类对象。

对于圆类来说,这意味着我们需要调用\verb|Point_ctor()|。就像所有动态绑定的方法一样,这个函数被声明为static并如此隐藏在Point.c中。然而,我们仍然可以通过类型描述符Point获取该函数,它在Circle.c中可用:

\begin{lstlisting}
	static void * Circle_ctor (void * _self, va_list * app) {
		struct Circle * self = 
			((const struct Class *) Point) -> ctor(_self, app);
		self->rad = va_arg(* app, int);
		return self;
	}
\end{lstlisting}

为什么我们传递app的地址而不是\verb|va_list|的值本身给每个构建子现在应该清楚了:\verb|new()|调用子类构建子,子类构建子又调用父类构建子,等等。最基本的构建子是第一个实际做事情的,并且它获取传递给\verb|new()|参数最左边的参数。剩下的参数留给接下来的子类,等等如此直到最后一个,最右边的参数被最终的子类取走,也就是说,通过\verb|new()|直接调用的构建子。

销毁过程最好是刚好相反的顺序:\verb|delete()|调用子类构建子。它应当只销毁自己的资源并接着调用它的直接父类析构子,可以销毁接着的资源等等。构建发生在子类比父类晚,析构则相反,子类在父类之前,圆部分在点部分之前。然而,这里并没有什么要做的。

我们之前有写过\verb|Circle_draw()|。我们使用可见成员并且编码表示如下:

\begin{lstlisting}
struct Point {
	const void * class;
	int x, y;	/* coordinates */
};
#define x(p)	(((const struct Point *) (p)) -> x)
#define y(p)	(((const struct Point *) (p)) ->y)
\end{lstlisting}

现在我们可以使用访问宏在\verb|Circle_draw()|中:

\begin{lstlisting}
static void Circle_draw (const void * _self) {
	const struct Circle * self = _self;
	
	printf("circle at %d,%d rad %d\n",
		x(self), y(self), self->rad);
\end{lstlisting}

\verb|move()|被静态绑定并且从点的实现中继承下来。我们推断出圆的实现,通过定义类型描述只在Circle.c中全局可见:

\begin{lstlisting}
static const struct Class _Circle = {
	sizeof(struct Circle), Circle_ctor, 0, Circle_draw
};

const void * Circle = & _Circle;
\end{lstlisting}

看起来我们有一个切实可行的策略来分发程序文本在类接口中的实现,表示和实现文件,圆和点的例子没有表现一个问题:如果一个动态绑定的方法例如\verb|Point_draw()|没有在子类中重写,子类描述符需要指向父类中函数的实现。而这里函数名被声明为static,所以选择子不能被避免。我们将会看到一个干净的解决方案给这个问题在第六章。作为一个临时方式,我们将避免使用static在这个情况下,声明函数头只在子类的实现文件中,并且使用这个函数名来初始化子类的类型描述。

\section{总结}
父类和子类的对象在行为上相似但不是相同。子类对象一般有更复杂和丰富的方法---它们是特殊化的父类对象。

我们从一份父类对象的表示的副本开始子类对象的表示,也就是说,一个子类对象通过添加成员到父类对象的末位来表示。

子类继承父类的方法:因为子类对象的开始部分看起开就是父类对象,我们可以向上类型转换并把指向子类对象的指针当作指向父类对象的指针传给父类方法。为了避免强制类型转换,我们声明所有的方法参数为\verb|void *|作为通用指针。

继承可以看作是多态的基本形势:父类方法接受不同类型的对象,也就是它本身的类型和所有子类。然而,因为对象都看起来象父类,这些方法仅仅作用在每个对象的父类部分,并且无差异的作用于不同的类。

动态绑定方法可以被子类继承或者重写---这取决于子类中使用了什么样的函数指针在类型描述中。因此,如果一个动态绑定的方法被一个对象调用,我们总是搜索属于对象真类的方法即使这个指针被向上转换到一些父类。如果一个动态绑定的方法被继承,它只能操作子类对象的父类部分,因为它不知道子类中有什么。如果一个方法被重写,子类版本可以访问整个对象,并且它可以调用它对象的父类方法通过显式的使用父类类型描述。

特别的,构建子应当调用父类构建子直到祖先,如此每个子类构建子只处理它自己的类扩展到它的父类表示。每个子类析构子应当移除子类的资源然后调用父类析构子等等直到祖先。构建从祖先到最后子类,析构顺序相反。

我们的策略有点问题,一般的我们不应该调用动态绑定的方法从构建子因为对象可能没有被完全初始化。在构建子调用之前,\verb|new()|插入最终类型描述到对象中。因此,如果一个构建子对一个对象调用一个动态绑定的方法,它将没有必要到达同一个类的方法作为构建子。安全的技术将会是构建子调用方法通过同一个类的内部名称,也就是说,对点来说调用\verb|Points_draw()|而不是\verb|draw()|。

为了鼓励信息隐藏,我们用三个文件实现一个类。接口文件包含抽象数据类型描述,表示文件包含对象的结构,实现文件包含方法和初始化类型描述的代码。一个接口文件包含父类接口文件并被实现文件和应用程序包含。表示文件包含父类表示文件并只被实现包含。

父类的成员不应当在子类直接被引用。相反的,我们可以提供静态绑定的访问和可能具有的修改方法对每个成员,或者我们可以添加适当的宏到父类的表示文件中。功能标记使得更容易的使用文本编辑器或者调试器来扫瞄可能的信息泄漏或者不变量的摧毁。

\section{是或者具有--继承vs聚集}

我们对于一个圆的表示包含了点的表示作为struct Circle的第一部分:

\begin{lstlisting}
struct Circle { const struct Point _; int rad; };
\end{lstlisting}

然而,我们志愿决定不直接访问这个成员。相反的,当我们想要继承,我们向上类型转换从Circle回到Point并且处理struct Point的初始化在那里。

这里有另一种表示圆的方式:它可以包含一个点作为聚集。我们可以只通过指针处理对象,也就是说,它不能从Point继承并重用它的方法。它可以使用点的方法到它的点的成员;它只是不能使用点方法到自身。

如果一个语言具有明确的继承语法,区别就更加明显了。相似的表示在C++中如下:

\begin{lstlisting}
struct Circle : Point { int rad; };	// inheritance

struct Cicle2 {
	struct Point point;	int rad;	// aggregate
};
\end{lstlisting}

在C++中我们没必要只作为指针访问对象。


继承,也就是说,从父类做一个子类,聚集,也就是说,包含一个对象作为其他对象的成员,提供了非常相似的功能。在一个特别的设计中使用那种途径可以通过是它或者具有它来测试:如果一个新类的对象只是像其他类的对象,我们应当使用继承来实现新类;如果一个新类的对象具有一个其他类的对象作为它的状态的一部分,我们应当使用聚集。

只要提到我们的点,一个圆只是一个大的点,这就是为什么我们使用继承在制作圆。一个矩形是一个不清楚的例子:我们可以通过一个引用点和边长来描述它,或者我们可以使用对角线或者三个角的终点。只有一个参考点是一个矩形角某种奇特的点;其他的表示导致聚集。在我们的算术表达式,我们可以使用继承来从一个一元到一个二元运算符节电,但是那将会违背测试。

\section{多重继承}

由于我们使用ANSI-C,所以我们不能隐藏继承的事实就是包含一个结构体在另一个的开始。向上类型转换是子类使用父类方法的关键。向上类型转换从圆回到点通过转换结构体开始的地址;地址的值没有发生变化。

如果我们在其他的结构体中包含两个或者更多的结构体,并且我们想要通过操作地址做一些向上类型转换,我们可以称这样的结果为多重继承:一个对象可以表现得如同它属于多个其他类。优点是我们不必要考虑非常小心的设计继承关系---我们可以快速的把类扔在一起并且继承需要的。缺点就是,明显的,在我们可以重用父类的方法之前,必须要进行地址操作。

事情很快就会变得迷惑不清。考虑一个文本和一个矩形,每个都有一个继承的引用点。我们可以把它们扔在一起作为一个按钮---唯一的问题是按钮应当继承一个还是两个引用点。C++允许任意一种方法步法在构建和向上类型转换中。

我们使用ANSI-C做每件事情的方式都有一个好处---它不会混淆继承的事实---多重或者其他---总是通过包含发生。包含,也可以通过聚集完成。并不是完全清楚多重继承给程序员帮助更多而不是复杂语言定义和增加实现负担。我们将保持事情的简单并且继续使用简单继承。第十四章中将会展示一个多重继承的原则,库合并,总是可以通过聚集和消息传递实现。

\section{练习}

图形编程提供了许多继承机会:一个点和一个边长定义一个正方形,一个点和一对偏移定义了一个矩形,一条线段,或者一个椭圆;一个点和一个数组偏移对定义一个多边形甚至花键。在我们处理所有这些类之前,我们可以制作更具智能的点通过添加文本和相关位置,或者通过引入颜色或者其他可视属性。

把\verb|move()|作为动态绑定是困难的,但是可能是有趣的:锁定的对象可以决定保持它们的点的引用不变并只移动它们的份额。

继承可以在许多领域中发现:集合,包和其他集合例如列表,栈,队列等等。字符串,原子和具有变量名和值的是其他的类。

父类可以被用于包装算数。如果我们假定存在动态绑定方法来比较和交换一个集合的元素基于一些正的索引值,我们可以实现一个父类包含一个排序算法。子类需要实现比较和交换它门的对象在数组中,但是它们继承排序的能力。



\chapter{编程经验--符号表}

一个结构体明智的加长,以此来共享基本结构的功能,可以帮助省去笨重的union的使用。特别在具有动态绑定,我们得到了一种统一的且完美健壮的方式处理消息传递。一旦基本机制就位,一个新的扩展类就可以重用基本代码容易的添加。

作为一个例子,我们将会添加关键字、常量、变量和数学函数到第三章开始的小计算器中。所有这些对象存在一个符号表中并且共享相同的名称搜索技术。

\section{扫瞄标识符}

在3.2节中我们实现了函数\verb|scan()|,从主程序获取一行输入并在每次调用返回一个输入符号。如果我们想引进关键字,命名常量等等,我们需要扩展\verb|scan()|。像浮点数一样,我们提取字符数字串以供深入分析:

\begin{lstlisting}

	#define ALNUM	"ABCDEFGHIJKLMNOPQRSTUVWXYZ" \
					"abcdefghijklmnopqrstuvwxyz" \
					"_" "0123456789"
	
	static enum tokens scan (const char * buf) {
		static const char * bp;
		…
		if (isdigit(* bp) || * bp == '.')
			…
		else if (isalpha( *bp) || *bp == '_') {
			char buf[BUFSIZ];
			int len = strspn(bp, ALNUM);
			
			if (len >= BUFSIZ)
				error("name too long: %-.10s…", bp);
			
			strncpy(buf, bp, len), buf[len] = '\0', bp += len;
			token = screen(buf);
		}
		...

\end{lstlisting}

一旦我们有了一个标识符,我们让新函数\verb|screen()|来决定它的token值应当是什么。如果有必要,\verb|screen()|将会存放一个解析器可以识别的符号描述到全局变量symbol中。

\section{使用变量}

一个变量参与两个操作:它的值被用作一个表达式的操作数,或者表达式的赋值对象。第一中操作是对3.5节中一个简单的\verb|factor()|扩展。

\begin{lstlisting}
static void * factor (void) {
	void * result;
	...
	switch (token) {
		case VAR:
			result = symbol;
			break;
		...
\end{lstlisting}

VAR是一个当\verb|screen()|发现适当的标识符的时候放到token中的唯一值。有关标识符的附加信息被放到全局变量symbol中。在这种情况下,symbol包含一个节点来标示变量作为表达式树中的一个叶子。\verb|screen()|要么找到变量再符号表中或者使用描述Var去创建它。

识别一个赋值有点复杂。如果我们的计算器允许如下两种语法的声明,它将会是舒适的:

\begin{lstlisting}
asgn : sum
	 | VAR = asgn
\end{lstlisting}

不幸的是,VAR也可以出现在sum的左端,也就是说,使用我们递归下降的技术如何识别C风格嵌入赋值不是立即就能清楚的。\footnote{这里有一个技巧:对sum做简单尝试。如果返回是下一个输入符号是=,sum必定是一个变量叶子节点,我们就可以创建这个赋值。}因为我们想要学到如何操作关键字,我们设置如下的语法:

\begin{lstlisting}
stmt  : sum
	| LET VAR = sum
\end{lstlisting}

这被翻译成如下函数:

\begin{lstlisting}
static void * stmt (void) {
	void * result;
	
	switch (token) {
		case LET:
			if (scan(0) != VAR)
				error("bad assignment");
			result = symbol;
			if (scan(0) != '=')
				error("expecting =");
			scan(0);
			return new(Assign, result, sum());
		default:
			return sum();
	} /* this i s comet */
}
\end{lstlisting}

在主程序中我们调用stmt()来替代sum(),并且我们的识别器已经准备好操作变量。Assign是一个新的类型描述,来计算一个sum的值并赋值给一个变量。

\section{筛子--Name}

赋值具有如下语法:

\begin{lstlisting}
stmt : sum
     | LET VAR = sum
\end{lstlisting}

LET是关键字的一个例子。在构建筛子的过程中我们仍然可以决定什么标识符将标示\verb|LET: scan()|从输入行提取一个标识符并传递给\verb|screen()|,是它在符号表中查找并返回token的适当值,至少一个变量,一个节点在symbol中。

识别器丢弃LET但是插入变量作为叶子节点在树中。对于另一个符号,例如一个算数函数的名称,我们可能想要适用\verb|new()|到screener返回的符号来获取一个新的节点。因此,我们的符号表入口应当对与大部分具有相同的函数动态绑定与我们树节点。

对于一个关键字,一个Name需要包含输入字符串和token值。稍后我们想要继承Name;因此,我们定义结构在Name.r中:

\begin{lstlisting}
struct Name {			/* base structure */
	const void * type;	/* for dynamic linkage */
	const char * name;	/* may be malloc-ed */
	int token;
};
\end{lstlisting}

我们的符号从不死亡:他们的名字是预定义的常量字符串还是存储的用户自定义变量动态字符串是没有关系的--我们将不会回收他们。

在我们可以定义一个符号之前,我们需要输入它到符号表。这不能通过调用\verb|new(Name, ...)|来处理,因为我们想要支持更多比Name复杂的符号,并且我们想要隐藏符号表的实现。相反的,我们提供一个函数\verb|install()|,它需要一个Name对象并把它插入到符号表中。这里给出符号表接口文件Name.h:

\begin{lstlisting}
extern void * symbol;	/* -> last Name found by screen() */
void install (const void * symbol);
int screen (const char * name);
\end{lstlisting}

识别器必须插入像LET的关键字到符号表中,在他们被screener发现之前。这些关键字可以被定义进一个常量表结构中--它对\verb|install()|没有影响。下面的函数被用来初始化识别:

\begin{lstlisting}
#include "Name.h"
#include "Name.r"

static void initName (void) {
	static const struct Name names [] = {
		{ 0, "let", LET },
		0
	};
	const struct Name * np;
	
	for (np = names; np->name; ++np)
		install(np);
}
\end{lstlisting}

注意\verb|names[]|,关键字表,不需要被存储。我们适用Name的标示来定义\verb|names[]|,也就是说,我们包含Name.r。由于关键字LET被丢弃,我们不提供动态绑定的方法。

\section{父类的实现--Name}

通过名字搜索符号是一个标准问题。不幸的是,ANSI标准没有定义一个合适的库函数来解决它。\verb|bsearch()|--有序表中的二分查找--比较接近,但是如果我们想要插入一个单独的新符号,我们不得不调用\verb|qsort()|来设置阶段给后续搜索。

UNIX系统很可能提供两三个函数家族来处理动态增长表。\verb|lsearch()|--线性搜索一个数组并在\verb|end(!)|添加--不是完全高校的。\verb|hsearch()|--一个结构体哈希表由一个文本和一个信息指针组成--维护一个固定大小的表并且强制一个尴尬的结构体入口。\verb|tsearch()|--一个二叉树具有任意比较和删除--是最常用的家族但是很没有效率,如果初始符号从一个有序序列中安装。

在一个UNIX系统上,\verb|tsearch()|有可能是最好的折衷。对于一个可移植的实现具有二元线程树可以在\verb|[Sch87]|找到。然而,如果这个家族不可用,或者如果我们不能保证一个随机的初始化,我们应当查看一个简单的设备来实现。一个小心实现的\verb|bsearch()|可以很容易的被扩展来支持存储的数组插入:

\begin{lstlisting}
void * binary (const void * key,
	void * _base, size_t * help, size_t width,
	int (* cmp) (const void * key, const void * elt)) {
	size_t nel = * nelp;
#define base	(* (char **) & _base)
	char * lim = base + nel * width, * high;

	if (nel > 0) {
		for (high = lim - width; base <= high; nel >>= 1) {
			char * mid = base + (nel >> 1) * width;
			int c = cmp(key, mid);

			if (c < 0)
				high = mid - width;
			else if (c > 0)
				base = mid + width, --nel;
			else
				return (void *) mid;
		}
\end{lstlisting}

到这里为止,这是一个任意数组的二元搜索。key只想要找的对象;base初始值是一个具有\verb|*nelp|个元素的表的开始位置,每个元素width字节;并且cmp是一个函数用来比较key和一个表中的元素。在此我们要么找到一个元素并返回它的位置,要么base现在是key应当出现在表中的位置地址。我们继续如下:

\begin{lstlisting}
		memmove(base + wdith, base, lim - base);
	}
	++ * nelp;
	return memcpy(base, key, width);
#undef base
}
\end{lstlisting}

\verb|memmove()|移动数组的末尾\footnote{\verb|memmove()|复制字节即使源和目标区域重叠;\verb|memcpy()|并不如此,但是它更具效率},\verb|memcpy()|插入key。我们假定数组之外还有空间并且我们通过nelp纪录我们已经加入了一个元素--\verb|binary()|和标准函数\verb|bsearch()|只需要地址而不是变量的值包含饿了表中元素的个数。

给出一个通用搜索和入口的方式,我们可以很轻易的管理我们的符号表。首先我们需要比较一个key和表中的元素:

\begin{lstlisting}
static int cmp (const void * _key, const void * _elt) {
	const char * const * key = _key;
	const struct Name * const * elt = _elt;

	return strcmp(* key, (* elt) -> name);
}
\end{lstlisting}

作为一个键值,我们只传递一个指向输入符号文本的指针的地址。表中的元素当然是Name 结构体,并且我们只查看他们的\verb|.name|成员。

搜索和入口通过适用适当的参数对\verb|binary()|进行调用来实现。由于我们事先不知道符号个数,我们确保一直有空间让我们扩招表:

\begin{lstlisting}
static struct Name ** search (const char ** name) {
	static const struct Name ** names;	/* dynamic table */
	static size_t used, max;

	if (used >= max) {
		names = names
			? realloc(names, (max *= 2) * sizeof * names)
			: malloc((max = NAMES) * sizeof * names);
		assert(names);
	}
	return binary(name, names, & used, sizeof * names, cmp);
}
\end{lstlisting}

NAMES是一个定义的常量具有初始分配的表项;每次我们用完,我们都是表的大小加倍。

\verb|search()|适用指向要查找的文本的地址指针作为参数并且返回表项的地址。如果文本未找到,\verb|binary()|就插入key--也就是说,只有指向文本的指针,而不是一个struct Name--到表中。这个策略是为了\verb|screen()|的利益,它只新建一个表元素,如果一个输入中的标识符是未知的:

\begin{lstlisting}
int screen (const char * name) {
	struct Name ** pp = search(& name);

	if (* pp == (void *) name)	/* entered name */
		* pp = new(Var, name);
	symbol = * pp;
	return (* pp) -> token;
}
\end{lstlisting}

\verb|screen()|让\verb|search()|查找要显示的输入符号。如果指符号文本的指针被插入符号表,我们需要使用一个新标识符的项目描述替换它。

对于\verb|screen()|,一个新的标识符必须是一个变量。我们假定这里有一个类型描述Var知道如何构建Name结构体来描述变量并且我们让\verb|new()|做剩下的工作。其他的情况,我们让symbol指向符号表项并且返回它的\verb|.token|值。

\begin{lstlisting}
void install (const void * np) {
	const char * name = ((struct Name *) np) -> name;
	struct Name ** pp = search(& name);

	if (* pp != (void *) name)
		error("cannot install name twice: %s", name);
	* pp = (struct Name *) np;
}
\end{lstlisting}

\verb|install()|比较简单。我们接受一个Name对象并且让\verb|search()|在符号表中找到它。\verb|install()|被假定来只处理新符号,所以我们应当总是能够插入对象替换它的名字。否则,如果\verb|search()|真的找到一个符号,我们就有麻烦了。

\section{子类的事先--Var}

\verb|screen()|调用\verb|new()|来创建一个新的变量符号并且返回它到识别器,并插入它到一个表达式树中。因此,Var必须创建可以项节点行为的符号表项,也就是说,当定义struct Var的时候,我们需要扩展一个struct Name来继承在符号表中存在的能力并且我们必须支持动态绑定的函数可以适用于表达式节点。我们描述接口在Var.h中:

\begin{lstlisting}
const void * Var;
const void * Assign;
\end{lstlisting}

一个变量具有一个名字和一个值。如果我们计算一个算术表达式的值,我们需要返回\verb|.value|成员。如果我们删除一个表达式,我们一定不能删除变量节点,因为它存活在符号表中:

\begin{lstlisting}
struct Var { struct Name _; double value; };
#define value(tree) (((struct Var *) tree) -> value)

static double doVar (const void * tree) {
	return value(tree);
}

static void freeVar (void * tree) {
}
\end{lstlisting}

就如在4.6节中讨论的,通过提供一个值的访问函数来简化代码。

创建一个变量需要分配一个struct Var,插入一个变量名的动态副本,并且标识值VAR被识别器规定:

\begin{lstlisting}
static void * mkVar (va_list ap) {
	struct Var * node = calloc(1, sizeof(struct Var));
	const char * name = va_arg(ap, const char *);
	size_t len = strlen(name);

	assert(node);
	node-> _.name = malloc(len + 1);
	assert(node -> _.name);
	strcpy((void *) node-> _.name, name);
	node -> _.token = VAR;
	return node;
}

static struct Type _Var = { mkVar, doVar, freeVar };

const void * Var = & _Var;

\verb|new()|照料插入Var类型描述到节点中,在符号被\verb|screen()|返回之前或者任何的使用。

就技术而言,\verb|mkVar()|是Name的构建子。然而,只有变量名需要被动态存储。因为饿哦我们决定在我们的计算器中构建子负责分配一个对象,我们不能让Var构建子调用一个Name构建子来维护\verb|.name|和\verb|.token|成员--一个Name构建子将会分配一个struct Name而不是一个struct Var。

\section{赋值}

赋值是一个二元操作。识别器保证我们具有一个变量作为做操作数和sum作为右操作数。因此,我们世纪需要实现的是实际赋值操作,也就是说,动态绑定进类型描述的\verb|.exec|成员:

\begin{lstlisting}
#include "value.h"
#include "value.r"

static double doAssign (const void * tree) {
	return value(left(tree)) = exec(right(tree));
}

static struct Type _Assign = { mkBin, doAssign, freeBin };
const void * Assign = & _Assign;
\end{lstlisting}

我们共享Bin的构建子和析构子,因此,在算数操作的实现中必须是全局的。我们也共享struct Bin和访问函数\verb|left()|和\verb|right()|。所有这些使用value.h导出并且实现文件value.r。我们自己的访问函数\verb|value()|对于struct Var故意的允许修改,如此赋值就可以被很优雅的实现。

\section{另一个子类--常量}

谁会喜欢输入\pi或者其他数学常量的值呢?我们从Kernighan和Pike\'s hoc\[K&P84\]得到线索并且预定义一些常量给我们的计算器。下面的函数需要被调用在初始化识别器期间:

\begin{lstlisting}
void initConst (void) {
	static const struct Var constants [] = {	/* like hoc */
		{ &_Var, "PI", CONST, 3.14159265358979323846 },
		...
		0 };
	const struct Var * vp;

	for (vp = constants; vp -> _.name; ++ vp)
		install(vp);
}
\end{lstlisting}

变量和常量几乎是一样的:都具有名称和值并且存活在符号表中;都返回他们的值在一个算数表达式的使用中;并且都不应当被删除,当我们删除一个算数表达式的时候。然而,我们不应当给常量赋值,所以我们需要同意一个新的标识符值CONST,识别器在\verb|factor()|中接受就像VAR一样,但是不允许在\verb|stmt()|的赋值的左边。

\section{数学函数--Math}

ANSI-C定义了许多数学函数例如sin(),sqrt(),exp()等等。作为另一个继承的练习,我们将添加库函数使用一个单个double参数并且具有一个double结构到我们的计算器。

这些函数工作的就如同一元运算符一样。我们可以定义一个新的类型给节点给每个函数并且收集大多数功能从Minus和Name类,但是这里有一个更简单的方法。我们扩展struct Name到struct Math如下:

\begin{lstlisting}
struct Math { struct Name _;
	double (* funct) (double);
};
#define funct(tree) (((struct Math *) left(tree)) -> funct)
\end{lstlisting}

额外的给函数名称用于输入和标识符给识别,我们存储像\verb|sin()|的库函数的地址在符号表项中。

在初始化期间我们调用下面的函数来输入所有的函数描述到符号表中:

\begin{lstlisting}
#include <math.h>

void initMath (void) {
	static const struct Math functions [] = {
		{ &_Math, "sqrt", MATH, sqrt },
		...
		0 };
	const struct Math * mp;

	for (mp = functions; mp -> _.name; ++ mp)
		install(mp);
}
\end{lstlisting}

一个函数调用是一个因子就好像使用一个减号标记一样。对于识别我们需要扩展我们的语法对因子:

\begin{lstlisting}
factor : NUMBER
     | - factor
     | ...
     | MATH ( sum )
\end{lstlisting}

MATH是公共标识符对所有函数输入通过\verb|initMath()|。这个翻译到下面的附加\verb|factor()|在识别器中:

\begin{lstlisting}
static void * factor (void) {
	void * result;
	...
	switch (token) {
		case MATH:
		{
			const struct Name * fp = symbol;

			if (scan(0) != '(')
				error("expecting (");
			scan(0);
			result = new(Math, fp, sum());
			if (token != ')')
				error("expecting )");
			break;
		}
\end{lstlisting}
symbol首先包含符号表元素对一个函数例如\verb|sin()|。我们保存这个指针并且构建表达式树对于函数参数通过调用\verb|sum()|。然后我们使用Math,类型描述给函数,并且让\verb|new()|构建下面的节点给表达式树:

TODO插入图片

我们让一个二元节点的左边只想符号表元素给函数并且我们附加参数树在右边。这个二元节点具有Math作为类型描述,也就是说,方法\verb|doMath()|和\verb|freeMath()|将会被调用来分别执行和删除节点。

Math节点仍然使用\verb|mkBin()|构建,因为这个函数不关心指针的后代。\verb|freeMath()|,然而,可能只会删除右子树:

\begin{lstlisting}
static void freeMath (void * tree) {
	delete(right(tree));
	free(tree);
}
\end{lstlisting}
如果我们仔细看上图,我们可以看到一个Math节点的执行是非常容易的。\verb|doMath()|需要调用存储在符号表中元素可以被访问的作为左后代二元节点从之被调用:

\begin{lstlisting}
static double doMath (const void * tree) {
	double result = exec(right(tree));

	errno = 0;
	result = funct(tree)(result);
	if (errno)
		error("error in %s: %s",
			((struct Math *) left(tree)) -> _.name,
			strerror(errno));
	return result;
}
\end{lstlisting}

唯一的问题是抓住数字错误通过检测errno变量在ANSI-C头文件errno.h中声明。这个完成了数学函数的实现给计算器。

\section{总结}

机遇一个函数\verb|binary()|来搜索和插入一个有序数组,我们实现了一个符号表包含了就诶够体具有名称和标识符值。继承允许我们插入其他结构体到表中而不需要改变函数搜索和插入。这种方式的高雅变得明显一旦我们考虑一个传统的定义一个符号表元素出于我们的目的:

\begin{lstlisting}
struct {
	const char * name;
	int token;
	union {			/* based on token */
		double value;
		double (* funct) (double);
	} u;
};
\end{lstlisting}

对于关键字,union是没有必要的。用户定义的函数讲会要求一个更详细的描述,并且引用union的部分是讨厌的。

继承允许我们适用符号表功能到新的项而不改变已经存在的代码。动态绑定在许多方式帮助保持实现的简单性:常量符号表元素,变量和函数可以被绑定进表达式树中而不用担心我们不慎删除他们;一个执行函数参考自身值有它自己安排节点。

\section{练习}

新关键字是必须的用来实现例如while或者repeat循环,if语句等等。识别被\verb|stmt()|处理,但是这个对于大部分情况,只有一个问题编译器构建,而不是继承。一旦我们决定语句描述,我们将创建节点类型例如While,Repeat或者IfElse,并且关键字在符号表中不需要知道他们的存在。

更有趣的是函数具有两个参数的例如\verb|atan2()|在ANSI-C数学库中。从这里看出符号表,这个函数被处理仅仅类似简单函数,但是对于表达式树我们需要开发一个新的节点类型使用三个后代。

用户定义的函数具有一个现实的有意思的问题。如果我们表示单独的参数通过\$并且我们使用一个节点类型Parm来只想函数项在符号表中从那里我们可以暂时存储参数值只要我们不允许递归,就是简单的。函数具有参数名和几个参数是比较困难的,当然了。然而,这是一个好的练习inverstigate继承的好处和动态绑定的好处。我们将在第11章中返回到这个问题。


\appendix

\end{document}
